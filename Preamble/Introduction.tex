\section*{Introducción}

Estos apuntes han sido elaborados por Pau Frangi Mahiques, Diego Rodríguez Cubero y Jaime Nieto Petinal, estudiantes de 3.º del Doble Grado en Matemáticas e Ingeniería Informática, con el objetivo de ofrecer un material claro, cohesionado y útil para el estudio de la asignatura de ASIGNATURA.

El contenido está basado en las clases del profesor/a NOMBRE PROFESOR, a quien agradecemos su dedicación y claridad expositiva.

El documento está organizado para acompañar el progreso del curso: DESCRIBIR CONTENIDOS. Se incluyen asimismo ejercicios seleccionados y exámenes resueltos, así como material gráfico y esquemas realizados con TikZ para apoyar la intuición geométrica.

Nuestro propósito es doble: por un lado, facilitar la preparación continuada de la asignatura mediante una exposición sistemática, con resultados claramente enunciados y, cuando procede, demostraciones o esbozos de demostración; por otro, servir como referencia rápida para técnicas de cálculo habituales. Aunque hemos procurado la máxima precisión, este material puede contener erratas u omisiones.

Invitamos encarecidamente a los estudiantes que cursen la asignatura en el futuro a ampliar, mejorar o complementar estos apuntes. Cualquier corrección, propuesta o contribución será bienvenida. Para ello, pueden:
\begin{itemize}
	\item Ponerse en contacto con nosotros mediante nuestros perfiles de GitHub, enlazados en la portada y en el pie de página.
	\item Podeis mandarnos un correo electrónico a:
	\begin{itemize}
        \item Diego Rodríguez Cubero: \href{mailto:diegorocux3x@gmail.com}{diegorocux3x@gmail.com}.
        \item Pau Frangi Mahiques: \href{mailto:pau.frangi@gmail.com}{pau.frangi@gmail.com}.
        \item Jaime Nieto Petinal: \href{mailto:jaimenietopetinal1@gmail.com}{jaimenietopetinal1@gmail.com}.
    \end{itemize}
    \item Abrir una incidencia (\emph{issue}) o enviar una contribución (\emph{pull request}) en el repositorio del proyecto.
\end{itemize}

Estos apuntes reflejan la materia cursada y quedan finalizados conforme al curso ya completado. A partir de aquí, invitamos a futuros alumnos a proponer mejoras, correcciones o ampliaciones; las contribuciones serán valoradas y, en su caso, incorporadas en ediciones posteriores. Agradecemos de antemano toda colaboración que ayude a hacerlos más claros, completos y útiles para la comunidad, ya sea puliendo detalles, corrigiendo erratas o añadiendo contenido relevante, ya sea teorico, ejemplos, ejercicios, examenes resueltos o material gráfico.