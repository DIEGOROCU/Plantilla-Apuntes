\begin{comment}
\end{comment}

\input{Exams/Examen 1.tex}
\newpage